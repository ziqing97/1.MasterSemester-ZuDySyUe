\chapter{Ausarbeitung}
\section{Aufgabe 1}
Der Zusammenhang zwischen gemessene Länge und gesuchte Länge für jede Epoche lautet:
\begin{gather}
	\overline{AB} + \overline{BC} = m_1 \\
	\overline{BC} + \overline{CD} = m_2 \\
	\overline{CD} + \overline{DE} = m_3 \\
	\overline{AB} + \overline{BC} + \overline{CD} = m_4 \\
	\overline{DE} = m_5
\end{gather}
Die Lösung der ersten Teilgleichung
\begin{equation}
	\bm{\hat{x}}(1) = (\bm{A^T} \bm{P} \bm{A})^{-1}  \bm{A^T}  \bm{P}  \bm{y_1} 
\end{equation}
\begin{gather}
\bm{e} = \bm{y} - \bm{A}\bm{x}(1) \\
\sigma(1) = \sqrt{\frac{\bm{e}' \bm{P} \bm{e}}{5-1}} \\
\bm{\Sigma}(\bm{\hat{x}}(1)) = \sigma^2(1)(\bm{A}^T \bm{P} \bm{A})^{-1}
\end{gather}
wobei:
\begin{gather}
\bm{y_1} = \begin{bmatrix}
m_1(1) \\
m_2(1) \\
m_3(1) \\
m_4(1) \\
m_5(1)
\end{bmatrix}, \quad 
\bm{A} =  \begin{bmatrix}
1 & 1 & 0 & 0\\
0 & 1 & 1 & 0\\
0 & 0 & 1 & 1\\
1 & 1 & 1 & 0\\
0 & 0 & 0 & 1 
\end{bmatrix}, \quad \bm{x} = \begin{bmatrix}
\overline{AB} \\
\overline{BC} \\
\overline{CD} \\
\overline{DE} 
\end{bmatrix} \quad \bm{P} = \begin{bmatrix}
\frac{1}{0,1^2} &  &  &  &  \\
 & \frac{1}{0,1^2} &  &  &  \\
 &  & \frac{1}{0,1^2} &  &  \\
 &  &  & \frac{1}{0,1^2} &  \\
 &  &  &  & \frac{1}{0,1^2}   
\end{bmatrix}
\end{gather}
Wenn man alle 8 Zeitpunkten berücksichtigt:
\begin{equation}
	\bm{y} = \begin{bmatrix}
	\bm{y_1} \\
	\bm{y_2} \\
	\bm{y_3} \\
	\vdots \\
	\bm{y_8} \\
	\end{bmatrix} \quad \bm{A_{sum}} = \begin{bmatrix}
	\bm{A} \\
	\bm{A} \\
	\bm{A} \\
	\vdots \\
	\bm{A} \\
	\end{bmatrix} \quad \bm{P_{sum}} = \begin{bmatrix}
	\bm{P} & \bm{0} & \cdots & \bm{0} \\
	\bm{0} & \bm{P} & \cdots & \bm{0} \\
	\vdots & \vdots & \ddots & \vdots \\
	\bm{0} & \bm{0} & \cdots & \bm{P}
	\end{bmatrix}
\end{equation}
Dann sind die sequentielle Lösung durch folgende Formeln gerechnet. Zu jeder Epoche die Berechnung ist unter Einbeziehung der Messungen aller vorangegangen Epochen. ($2 \leq i \leq 8$)
\begin{gather}
	\bm{\hat{x}}(i) = \bm{\hat{x}}(i-1) + \left[\sigma(i-1)^2 (\bm{\Sigma}(\bm{\hat{x}}(i-1)))^{-1} + \bm{A}^T \bm{P} \bm{A}\right]^{-1} \bm{A}^T \bm{P} (\bm{y_i} - \bm{A} \bm{\hat{x}}(i-1)) \\
	\sigma(i) = \sqrt{\frac{1}{r(i)} (r(i-1) + \Delta\bm{\hat{x}}^T \bm{\Sigma}(\bm{\hat{x}}(i-1)) \Delta \bm{\hat{x}}) + \bm{e_i}^T \bm{P} \bm{e_i}} \\
	\bm{\Sigma}(\bm{\hat{x}}(i)) = \sigma^2(i) (\sigma^2(i-1) \bm{\Sigma}(\bm{\hat{x}}(i-1)) + \bm{A}^T \bm{P} \bm{A})^{-1} \\
	\bm{e_i} = \bm{y_i} - \bm{A}\bm{\hat{x}} \\
	r(i) = 4i-5
\end{gather}
Die Abstände und Fehler von $t_1$ bis $t_8$ sind: 
\begin{table}[htbp]\centering
	\begin{tabular}{|l|l|l|l|l|l|}
		\hline
		& $\overline{AB}$ \ut{m}   & $\overline{BC}$ \ut{m}   & $\overline{CD}$ \ut{m}   & $\overline{DE}$ \ut{m}   & $\sigma$ \ut{m} \\ \hline
		$t_1$ & 1,76 & 0,86 & 2,10 & 1,49 & 1,25  \\ \hline
		$t_2$ & 1,75 & 0,97 & 2,00 & 1,61 & 1,29  \\ \hline
		$t_3$ & 1,61 & 1,10 & 1,91 & 1,56 & 1,65  \\ \hline
		$t_4$ & 1,64 & 1,13 & 1,85 & 1,59 & 1,65  \\ \hline
		$t_5$ & 1,63 & 1,16 & 1,84 & 1,61 & 1,47  \\ \hline
		$t_6$ & 1,61 & 1,18 & 1,84 & 1,61 & 1,33  \\ \hline
		$t_7$ & 1,59 & 1,20 & 1,83 & 1,60 & 1,24  \\ \hline
		$t_8$ & 1,58 & 1,20 & 1,82 & 1,61 & 1,16  \\ \hline
	\end{tabular}
	\caption{Für erste 8 Zeitpunkten}
\end{table}\\
Die Ergebnisse für die andere 8 Epochen:
\begin{table}[htbp]\centering
	\begin{tabular}{|l|l|l|l|l|l|}
		\hline
		& $\overline{AB}$ \ut{m}   & $\overline{BC}$ \ut{m}   & $\overline{CD}$ \ut{m}   & $\overline{DE}$ \ut{m}   & $\sigma$ \ut{m} \\ \hline
		$t_1$ & 1,48 & 1,34 & 1,78 & 1,55 & 0,15  \\ \hline
		$t_2$ & 1,49 & 1,31 & 1,78 & 1,58 & 0,32  \\ \hline
		$t_3$ & 1,50 & 1,32 & 1,75 & 1,62 & 0,48  \\ \hline
		$t_4$ & 1,52 & 1,29 & 1,95 & 1,62 & 2,88  \\ \hline
		$t_5$ & 1,52 & 1,30 & 2,06 & 1,61 & 3,19  \\ \hline
		$t_6$ & 1,51 & 1,30 & 2,14 & 1,61 & 3,17  \\ \hline
		$t_7$ & 1,51 & 1,30 & 2,18 & 1,63 & 3,10  \\ \hline
		$t_8$ & 1,52 & 1,29 & 2,23 & 1,63 & 3,01  \\ \hline
	\end{tabular}
	\caption{Für zweite 8 Epochen}
\end{table}\\
Bei den erst 8 Messungen sind $\sigma$ relativ groß aber konstant weil die Leute nicht ruhig bleiben aber sie bewegen sich auch nicht. Bei den zweit Messungen ist $\sigma$ seit $t_4$ erhöht, das ist die Zeitpunkt wenn der Person sich bewegt hat. 
\clearpage
\section{Aufgabe 2}
\section{Aufgabe 3}
Die Runge-Kutta dient um die Differentialgleichungen zu lösen, hier soll $y' = t^2 + 2t - y + 1$ mit Anfangswert $y(0) = 0$ an der Stelle $t=0.5$ berechnet werden. Die Schrittweite $h = 0.1$.\\\\
Runge-Kutta dritter Ordnung:
\begin{align}
	y_{n+1} = & \  y_n + \frac{h}{6} (k_1 + 4k_2 + k_3) \\
	k_1 = & \  f(y_n,t_n) \\
	k_2 = & \  f(y_n + \frac{h}{2}k_1, t_n + \frac{h}{2}) \\
	k_3 = & \  f(y_n - hk_1 + 2hk_2, t_n + h) \\
	\text{mit} & \quad  f(t,y_n) = y_n'
\end{align}
Die Ergebnisse für jede Schritt:
\begin{table}[htpb]\centering
	\begin{tabular}{|c|c|c|c|c|c|c|}
		\hline
		n & 0 & 1  & 2  & 3  & 4  & 5  \\ \hline
		t &0     & 0,1    & 0,2    & 0,3    & 0,4    & 0,5    \\ \hline
		y & 0     & 0,1052 & 0,2213 & 0,3492 & 0,4897 & 0,6434 \\ \hline
	\end{tabular}
	\caption{Dritte RK Verfahren}
\end{table}\\
$y(0,5) = 0,6434$ nach Runge Kutta dritter Ordnung. \\\\
analog, Runge-Kutta vierter Ordnung: 
\begin{align} \label{2}
	y_{n+1} = & \  y_n + \frac{h}{6} (k_1 + 2k_2 +2 k_3 + k_4) \\
	k_1 = & \  f(y_n,t_n) \\
	k_2 = & \  f(y_n + \frac{h}{2}k_1, t_n + \frac{h}{2}) \\
	k_3 = & \  f(y_n + \frac{h}{2}k_2, t_n + \frac{h}{2}) \\
	k_4 = & \  f(y_n + hk_3,t_n + h) \label{3}
\end{align}
\begin{table}[htpb]\centering
	\begin{tabular}{|c|c|c|c|c|c|c|}
		\hline
		n & 0 & 1  & 2  & 3  & 4  & 5  \\ \hline
		t &0     & 0,1    & 0,2    & 0,3    & 0,4    & 0,5    \\ \hline
		y & 0     & 0,1052 & 0,2213 & 0,3492 & 0,4897 & 0,6435 \\ \hline
	\end{tabular}
	\caption{Vierte RK Verfahren}
\end{table}\\
Die Unterschied zwischen dritte und vierte Ordnung an $t = 0,5$ ist $-2,14 \cdot 10^{-5}$. Das ist sehr klein und kann in meisten Fälle ignoriert werden.  \clearpage
\section{Aufgabe 4}
Die Differentialgleichung ist von $y$ und $c$ unabhängig:
\begin{equation}\label{1}
	y' = c
\end{equation}
In dieser Aufgabe ist $y_{n+m}$ zu berücksichtigen. Nach dem Einsatz von \ref{1}:
\begin{align}
	y_{n+m} & \ = y_{n+m-1} + hc \\
	& \ = y_{n+m-2} +hc+hc \\
	 & \quad \cdots  \\
	 & \ = y_n + mhc
\end{align}
Weil $y_n$ und $c$ unkorreliert sind, lautet die Fehlerfortpflanzung
\begin{gather}
	\frac{\partial y_{n+m}}{\partial y_n} = 1 \\
	\frac{\partial y_{n+m}}{\partial c} = mh \\
	\sigma^2_{y_{m+n}} = \begin{bmatrix}
	1 & mh
	\end{bmatrix} \cdot \begin{bmatrix}
	\sigma^2_{y_n} & 0 \\
	0 & \sigma^2_{c}
	\end{bmatrix} \cdot \begin{bmatrix}
	1 \\
    mh
	\end{bmatrix} = \sigma^2_{y_n} + (mh)^2\sigma^2_{c}\\
	\sigma_{y_n} = \sqrt{\sigma^2_{y_n} + (mh)^2\sigma^2_{c}}
\end{gather}
wobei $\sigma_{y_n}$ und $\sigma_{c}$ sind die Unsicherheit von $y_n$ und $c$. \clearpage
\section{Aufgabe 5}
\subsection{a}\label{seca}
In dieser Teilaufgabe ist die Koordinaten mit Runge-Kutta Verfahren vierter Ordnung mit Schrittweite $h = \SI{100}{s}$ von $t_1 = \SI{1000}{s}$ nach $t_s = \SI{1900}{s}$ berechnet. Die Formeln sind in \ref{2} bis \ref{3}. 
\begin{table}[htbp] \centering
	\begin{tabular}{|l|l|}
		\hline
		$x$ (m)     & 1.880289566568e+07  \\ \hline
		$y$ (m)     & 1.661568254689e+07  \\ \hline
		$z$ (m)     & -4.599055621492e+06 \\ \hline
		$v_x$ (m/s) & -4.070049594214e+02  \\ \hline
		$v_y$ (m/s) & -5.159170009037e+02 \\ \hline
		$v_z$ (m/s) & -3.503303895904e+03 \\ \hline
	\end{tabular}
	\caption{Position und Geschwindigkeit an 1900s (von 1000s mit Schrittweite 100s)}
\end{table}
\subsection{b}\label{secb}
Ähnlich wie \ref{seca}, aber von $t_2 = \SI{2800}{s}$ nach $t_s$:
\begin{table}[htbp]\centering
	\begin{tabular}{|l|l|}
		\hline
		$x$ (m)     & 1.880289473488e+07  \\ \hline
		$y$ (m)     & 1.661568259796e+07  \\ \hline
		$z$ (m)     & -4.599055761960e+06 \\ \hline
		$v_x$ (m/s) & -4.070061798922e+02 \\ \hline
		$v_y$ (m/s) & -5.159183580348e+02 \\ \hline
		$v_z$ (m/s) & -3.503303232403e+03 \\ \hline
	\end{tabular}
	\caption{Position und Geschwindigkeit an 1900s (von 2800s mit Schrittweite 100s)}
\end{table}
\subsection{c}\label{secc}
\ref{seca} und \ref{secb} werden wiederholt mit Schrittweite $h = \SI{1}{s}$:
\begin{table}[htpb]\centering
	\begin{tabular}{|l|l|}
		\hline
		$x$ (m)     & 1.880289566450e+07  \\ \hline
		$y$ (m)     & 1.661568254808e+07  \\ \hline
		$z$ (m)     & -4.599055623060e+06 \\ \hline
		$v_x$ (m/s) & -4.070049591118e+02 \\ \hline
		$v_y$ (m/s) & -5.159170004552e+02 \\ \hline
		$v_z$ (m/s) & -3.503303895892e+03 \\ \hline
	\end{tabular}
	\caption{Position und Geschwindigkeit an 1900s (von 1000s mit Schrittweite 1s)}
\end{table}\clearpage
\begin{table}[htbp]\centering
	\begin{tabular}{|l|l|}
		\hline
		$x$ (m)     & 1.880289473654e+07  \\ \hline
		$y$ (m)     & 1.661568259745e+07  \\ \hline
		$z$ (m)     & -4.599055760315e+06 \\ \hline
		$v_x$ (m/s) & -4.070061799650e+02 \\ \hline
		$v_y$ (m/s) & -5.159183586592e+02 \\ \hline
		$v_z$ (m/s) & -3.503303232411e+03 \\ \hline
	\end{tabular}
	\caption{Position und Geschwindigkeit an 1900s (von 2800s mit Schrittweite 1s)}
\end{table}
\subsection{d}
In dieser Teilaufgabe werden \ref{seca}, \ref{secb} und \ref{secc} mit Runge-Kutta zweiter Ordnung statt Runge-Kutta vierter Ordnung wiederholt. 
\begin{align}
	y_{n+1} & \ = y_n + \frac{h}{2}(k_1 + k_2) \\
	k_1 & \ = f(y_n,t_n) \\
	k_2 & \ = f(y_n + hk_1, t_n + h)
\end{align}
Die Ergebnisse:
\begin{table}[htbp]\centering
	\begin{tabular}{|l|l|}
		\hline
		$x$ (m)     & 1.880289566568e+07  \\ \hline
		$y$ (m)     & 1.661568254689e+07  \\ \hline
		$z$ (m)     & -4.599055621492e+06 \\ \hline
		$v_x$ (m/s) & -4.070049594214e+02 \\ \hline
		$v_y$ (m/s) & -5.159170009037e+02 \\ \hline
		$v_z$ (m/s) & -3.503303895904e+03 \\ \hline
	\end{tabular}
	\caption{Position und Geschwindigkeit an 1900s (von 1000s mit Schrittweite 100s)(RK2)}
\end{table}\\
\begin{table}[htbp]\centering
	\begin{tabular}{|l|l|}
		\hline
		$x$ (m)     & 1.880289473488e+07  \\ \hline
		$y$ (m)     & 1.661568259796e+07  \\ \hline
		$z$ (m)     & -4.599055761960e+06 \\ \hline
		$v_x$ (m/s) & -4.070061798922e+02 \\ \hline
		$v_y$ (m/s) & -5.159183580348e+02 \\ \hline
		$v_z$ (m/s) & -3.503303232403e+03 \\ \hline
	\end{tabular}
	\caption{Position und Geschwindigkeit an 1900s (von 2800s mit Schrittweite 100s)(RK2)}
\end{table}
\clearpage
\begin{table}[htbp]\centering
	\begin{tabular}{|l|l|}
		\hline
		$x$ (m)     & 1.880289566450e+07  \\ \hline
		$y$ (m)     & 1.661568254808e+07  \\ \hline
		$z$ (m)     & -4.599055623060e+06 \\ \hline
		$v_x$ (m/s) & -4.070049591118e+02 \\ \hline
		$v_y$ (m/s) & -5.159170004552e+02 \\ \hline
		$v_z$ (m/s) & -3.503303895892e+03 \\ \hline
	\end{tabular}
	\caption{Position und Geschwindigkeit an 1900s (von 1000s mit Schrittweite 1s)(RK2)}
\end{table}
\begin{table}[htbp]\centering
	\begin{tabular}{|l|l|}
		\hline
		$x$ (m)     & 1.880289473654e+07  \\ \hline
		$y$ (m)     & 1.661568259745e+07  \\ \hline
		$z$ (m)     & -4.599055760315e+06 \\ \hline
		$v_x$ (m/s) & -4.070061799650e+02 \\ \hline
		$v_y$ (m/s) & -5.159183586592e+02 \\ \hline
		$v_z$ (m/s) & -3.503303232411e+03 \\ \hline
	\end{tabular}
	\caption{Position und Geschwindigkeit an 1900s (von 2800s mit Schrittweite 1s)(RK2)}
\end{table}
\subsection{e}
